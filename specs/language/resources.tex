\Ch{Resource Types}{Resources}

Resources are built-in intangible types representing memory with an external interface.

These take the form of Typed Buffers, Raw Buffers, Textures, and Samplers.
Buffer and Texture types can be read-only or writable.

\Sec{Typed Buffers}{Resources.tybufs}

The typed buffer class template represents a one-dimensional resource containing an array of a single given type.
Its contents are indexed by typed access to each element
Types may have their formats converted upon load.

Template types can be any type that totals or can be padded to 16 bytes.

All typed buffers can be read through subscript operators or Load methods.
Writable typed buffers can be written through subscript operators.

Where \texttt{T} can be any arithmetic type \ref{Basic.types.arithmetic}
or a vector with a maximum of four elements containing such an arithmetic type.
The total size of a single contained element must be less than 16 bytes.

Typed buffers perform format conversions on load such that the underlying data
gets converted to the destination type.

\begin{HLSL}
template <typename T = float4>
 class Buffer {
 public:
   Buffer();
   Buffer(const Buffer &buf);
   Buffer operator=(Buffer buf);

   void GetDimensions(out uint width) const;

   // Element access.
   T operator[](int index) const;
   T Load(in int index) const;
   T Load(in int index, out uint status) const;
};


template <typename T>
 class RWBuffer : public Buffer {
 public:
   RWBuffer();
   RWBuffer(RWBuffer buf);
   RWBuffer operator=(RWBuffer buf);

   // element access
   T &operator[](int n);
};
}
\end{HLSL}

\Sub{Constructors}{Resources.tybufs.ctrs}

Read-only Buffers can be defined with explicit or implicit template parameter types.
\begin{HLSL}
  Buffer buf1;
  Buffer<> buf2;
  Buffer<float4> buf3;
\end{HLSL}
Since \texttt{float4} is the default type, all of these definitions are equivalent.

RWBuffers must be defined with explicit template parameter types.
\begin{HLSL}
  RWBuffer<float4> buf;
\end{HLSL}

When defined at global scope, typed buffers are bound to externally-defined backing stores
using the explicit binding location if provided or the implicit binding if not (\ref{Resources.binding}).

When defined at local scope, typed buffers represent local references
that can be associated with global buffers when assigned,
but must be resolvable to a unique global buffer declaration.

\begin{note}
Local buffers must be resolvable to a unique global declaration, but not necessarily a unique individual buffer.
Local buffer assignments can map to different resource array elements depending on runtime values.
\end{note}

\begin{HLSL}
  Buffer<int> grobuf;
  RWBuffer<int> grwbuf;
  void main() {
    Buffer<int> lrobuf = grobuf;
    RWBuffer<int> lrwbuf = grwbuf;
  }
\end{HLSL}
Buffer operands to the assignment operator must have the same element type.

\Sub{Dimensions}{Resources.tybufs.dims}

The size of a typed buffer can be retrieved using the \texttt{GetDimensions} method.
\begin{HLSL}
void GetDimensions(out uint width) const;
\end{HLSL}

This returns the full length, in elements of the buffer through the \texttt{width} out parameter.

\Sub{Element Access}{Resources.tybufs.access}

The contents of typed buffers can be retrieved using the \texttt{Load} methods
or the subscript operator.

\begin{HLSL}
 T Load(in int index) const;
 T operator[](int index) const;
\end{HLSL}

These each return the element of the buffer's type at the given index \texttt{index}.

An additional \texttt{Load} method returns the indicated element just as the other,
but also takes a second \texttt{status} parameter that returns information about the accessed resource.
\begin{HLSL}
 T Load(in int index, out uint status) const;
\end{HLSL}

The \texttt{status} parameter returns an indication of whether all of the retrieved value
came from fully mapped parts of the resource.
This parameter must be passed to the built-in function \texttt{CheckAccessFullyMapped} (\ref{Resources.mapcheck})
in order to interpret it.

Writable buffers have an additional subscript operator that allows assignment to an element of the buffer.
It behaves as if it returns a reference to that element.
\begin{HLSL}
 T &operator[](int index);
\end{HLSL}

Partial writes aren't allowed.
Assignment to individual vector elements will result in an error.

\Sec{Raw Buffers}{Resources.rawbufs}

Raw buffers are one-dimensional resources of arbitrary memory layout.
They are either ByteAddressBuffers or StructuredBuffers.
ByteAddressBuffers enable per-byte access to the raw data while
StructuredBuffers have an associated structure type that determines how they are
indexed. This type can be a scalar, vector, matrix, or user-defined struct.

\Sec{Byte Access Buffers}{Resources.babufs}

\begin{HLSL}
 class ByteAddressBuffer {
 public:
   ByteAddressBuffer();
   ByteAddressBuffer(const ByteAddressBuffer &buf);
   ByteAddressBuffer operator=(ByteAddressBuffer buf);

   void GetDimensions(out uint size) const;

   // Element access.
   uint Load(in uint offset) const;
   uint2 Load2(in uint offset) const;
   uint3 Load3(in uint offset) const;
   uint4 Load4(in uint offset) const;
   template<typename T>
   T Load(in uint offset) const;

   uint Load(in uint offset, out uint status) const;
   uint2 Load2(in uint offset, out uint status) const;
   uint3 Load3(in uint offset, out uint status) const;
   uint4 Load4(in uint offset, out uint status) const;
   template<typename T>
   T Load(in uint offset, out uint status) const;
};

 class RWByteAddressBuffer : public ByteAddressBuffer {
 public:
   RWByteAddressBuffer();
   RWByteAddressBuffer(const RWByteAddressBuffer &buf);
   RWByteAddressBuffer operator=(RWByteAddressBuffer buf);

   // Element assignment.
   void Store(in uint offset, in uint value);
   void Store2(in uint offset, in uint2 value);
   void Store3(in uint offset, in uint3 value);
   void Store4(in uint offset, in uint4 value);
   template<typename T>
   void Store(in uint offset, in T value);

   // 32-bit integer atomic arithmetic/bitwise operations.
   void InterlockedAdd(in uint offset, in uint value);
   void InterlockedAdd(in uint offset, in uint value, out uint original);
   void InterlockedAnd(in uint offset, in uint value);
   void InterlockedAnd(in uint offset, in uint value, out uint original);
   void InterlockedOr(in uint offset, in uint value);
   void InterlockedOr(in uint offset, in uint value, out uint original);
   void InterlockedXor(in uint offset, in uint value);
   void InterlockedXor(in uint offset, in uint value, out uint original);

   // 32-bit integer atomic comparison operations
   void InterlockedMin(in uint offset, in int value);
   void InterlockedMin(in uint offset, in uint value);
   void InterlockedMin(in uint offset, in int value, out int original);
   void InterlockedMin(in uint offset, in uint value, out uint original);
   void InterlockedMax(in uint offset, in int value);
   void InterlockedMax(in uint offset, in uint value);
   void InterlockedMax(in uint offset, in int value, out int original);
   void InterlockedMax(in uint offset, in uint value, out uint original);

   // 32-bit integer atomic compare/exchange operations
   void InterlockedCompareStore(in uint offset, in uint compare, in uint value);
   void InterlockedExchange(in uint offset, in uint value, out uint original);
   void InterlockedCompareExchange(in uint offset, in uint compare, in uint value,
                                    out uint original);

   // 64-bit integer atomic arithmatic/bitwise operations
   void InterlockedAdd64(in uint offset, in uint64_t value);
   void InterlockedAdd64(in uint offset, in uint64_t value, out uint64_t original);
   void InterlockedAnd64(in uint offset, in uint64_t value);
   void InterlockedAnd64(in uint offset, in uint64_t value, out uint64_t original);
   void InterlockedOr64(in uint offset, in uint64_t value);
   void InterlockedOr64(in uint offset, in uint64_t value, out uint64_t original);
   void InterlockedXor64(in uint offset, in uint64_t value);
   void InterlockedXor64(in uint offset, in uint64_t value, out uint64_t original);

   // 64-bit integer atomic comparison operations
   void InterlockedMin64(in uint offset, in int64_t value);
   void InterlockedMin64(in uint offset, in int64_t value, out int64_t original);
   void InterlockedMin64(in uint offset, in uint64_t value);
   void InterlockedMin64(in uint offset, in uint64_t value, out uint64_t original);
   void InterlockedMax64(in uint offset, in int64_t value);
   void InterlockedMax64(in uint offset, in int64_t value, out int64_t original);
   void InterlockedMax64(in uint offset, in uint64_t value);
   void InterlockedMax64(in uint offset, in uint64_t value, out uint64_t original);

   // 64-bit integer atomic compare/exchange operations
   void InterlockedCompareStore64(in uint offset, in uint64_t compare,
                                   in uint64_t value);
   void InterlockedExchange64(in uint offset, in int64_t value,
                               out int64_t original);
   void InterlockedCompareExchange64(in uint offset, in uint64_t compare,
                                      in uint64_t value, out int64_t original);

   // 32-bit float atomic compare/exchange operations
   void InterlockedCompareStoreFloatBitwise(in uint byteOffest,
                                             in float compare, in float value);
   void InterlockedExchangeFloat(in uint byteOffest, in float value,
                                  out float original);
   void InterlockedCompareExchangeFloatBitwise(in uint byteOffest,
                                                in float compare,
                                                in float value,
                                                out float original);
};
\end{HLSL}

\Sub{Constructors}{Resources.babufs.ctrs}

Since their contents are viewed as raw bytes, ByteAddressBuffers are defined without any type parameters.
\begin{HLSL}
  ByteAddressBuffer robuf;
  RWByteAddressBuffer rwbuf;
\end{HLSL}

When defined at global scope, ByteAccessBuffers are bound to externally-defined backing stores
using the explicit binding location if provided or the implicit binding if not (\ref{Resources.binding}).

When defined at local scope, ByteAccessBuffers represent local references
that can be associated with global ByteAccessBuffers when assigned,
but must be resolvable to a unique global buffer declaration.

\begin{HLSL}
  ByteAddressBuffer grobuf;
  RWByteAddressBuffer grwbuf;
  void main() {
    ByteAddressBuffer lrobuf = grobuf;
    RWByteAddressBuffer lrwbuf = grwbuf;
  }
\end{HLSL}

\Sub{Dimensions}{Resources.babufs.dims}

The size of a ByteAddressBuffer can be retrieved using the \texttt{GetDimensions} method.
\begin{HLSL}
void GetDimensions(out uint size) const;
\end{HLSL}

This returns the full size of the buffer in bytes through the \texttt{size} out parameter.

\Sub{Element Access}{Resources.babufs.access}

The contents of ByteAddressBuffers can be retrieved using the \texttt{Load} methods.

\begin{HLSL}
 uint Load(in uint offset) const;
 uint2 Load2(in uint offset) const;
 uint3 Load3(in uint offset) const;
 uint4 Load4(in uint offset) const;
\end{HLSL}

These each return the bytes at the given byte offset into the buffer \texttt{offset} in the form of one or more unsigned integers.
The \texttt{offset} address must be a multiple of 4.
Each of the variants returns a number of bytes corresponding to the size of the uint vector returned.

An additional templated load method allows returning the bytes at the given byte offset in the form
of the type given in the template parameter. This can be a scalar, vector, matrix, or user-defined struct.

\begin{HLSL}
  template<typename T>
  T Load(in uint offset) const;
\end{HLSL}

The alignment requirements of \texttt{offset} for this operation is the size in bytes of the largest
scalar type contained in the given tyep \texttt{T}.

Additional \texttt{Load} methods return the same values as the others,
but also take a second \texttt{status} parameter that returns information about the accessed resource.
\begin{HLSL}
  uint Load(in uint offset, out uint status) const;
  uint2 Load2(in uint offset, out uint status) const;
  uint3 Load3(in uint offset, out uint status) const;
  uint4 Load4(in uint offset, out uint status) const;
  template<typename T>
  T Load(in uint offset, out uint status) const;
\end{HLSL}

The \texttt{status} parameter returns an indication of whether all of the retrieved value
came from fully mapped parts of the resource.
This parameter must be passed to the built-in function \texttt{CheckAccessFullyMapped} (\ref{Resources.mapcheck})
in order to interpret it.

\Sub{Atomic Operations}{Resources.babufs.atomics}

RWByteAddressBuffers have a suite of atomic methods that perform the given operation
on type-appropriate-sized element data at the given buffer offset
in a way that ensures no contention from other threads performing other operations.
Each of these operates on two 32 or 64 bit values: the given parameter value and the buffer value.
The buffer value is a 32 or 64 bit value at a given offset into the buffer.
Each method has an overload that returns the original buffer value before the atomic operation was performed.

\begin{HLSL}
   void InterlockedAdd(in uint offset, in uint value);
   void InterlockedAdd(in uint offset, in uint value, out uint original);
   void InterlockedAnd(in uint offset, in uint value);
   void InterlockedAnd(in uint offset, in uint value, out uint original);
   void InterlockedOr(in uint offset, in uint value);
   void InterlockedOr(in uint offset, in uint value, out uint original);
   void InterlockedXor(in uint offset, in uint value);
   void InterlockedXor(in uint offset, in uint value, out uint original);
\end{HLSL}

Perform atomic addition, bitwise and, bitwise or, or bitwise exclusive or on
the 32-bit integer buffer value at the byte offset \texttt{offset} and the provided \texttt{value}.
Store the result to that offset buffer location,
optionally returning the original buffer value through \texttt{original}.

\begin{HLSL}
   void InterlockedMin(in uint offset, in int value);
   void InterlockedMin(in uint offset, in uint value);
   void InterlockedMin(in uint offset, in int value, out int original);
   void InterlockedMin(in uint offset, in uint value, out uint original);
   void InterlockedMax(in uint offset, in int value);
   void InterlockedMax(in uint offset, in uint value);
   void InterlockedMax(in uint offset, in int value, out int original);
   void InterlockedMax(in uint offset, in uint value, out uint original);
\end{HLSL}

Perform the sign-approrpriate minimum or maximum comparison on
the 32-bit integer buffer value at the byte offset \texttt{offset} and the provided \texttt{value}.
Store the lesser or greater value, as appropriate, to that offset buffer location,
optionally returning the original buffer value through \texttt{original}.

\begin{HLSL}
   void InterlockedExchange(in uint offset, in uint value, out uint original);
\end{HLSL}

Performs an atomic exchange of
the 32-bit integer buffer value at the byte offset \texttt{offset} with the provided \texttt{value}.
Stores \texttt{value} at the offset buffer location
and returns the original buffer value through \texttt{original}.

\begin{HLSL}
   void InterlockedCompareStore(in uint offset, in uint compare, in uint value);
   void InterlockedCompareExchange(in uint offset, in uint compare, in uint value,
                                    out uint original);
\end{HLSL}

Perform an atomic exchange of or store to
the 32-bit integer buffer value at the byte offset \texttt{offset} with the provided \texttt{value}
if the value at that location matches the value of \texttt{compare}.
\texttt{InterlockedCompareExchange} returns the original buffer value through \texttt{original}.

\begin{HLSL}
   void InterlockedAdd64(in uint offset, in uint64_t value);
   void InterlockedAdd64(in uint offset, in uint64_t value, out uint64_t original);
   void InterlockedAnd64(in uint offset, in uint64_t value);
   void InterlockedAnd64(in uint offset, in uint64_t value, out uint64_t original);
   void InterlockedOr64(in uint offset, in uint64_t value);
   void InterlockedOr64(in uint offset, in uint64_t value, out uint64_t original);
   void InterlockedXor64(in uint offset, in uint64_t value);
   void InterlockedXor64(in uint offset, in uint64_t value, out uint64_t original);
\end{HLSL}

Perform atomic addition, bitwise and, bitwise or, or bitwise exclusive or on
the 64-bit integer buffer value at the byte offset \texttt{offset} and the provided \texttt{value}.
Store the result to that offset buffer location,
optionally returning the original buffer value through \texttt{original}.

\begin{HLSL}
   void InterlockedMin64(in uint offset, in int64_t value);
   void InterlockedMin64(in uint offset, in uint64_t value);
   void InterlockedMin64(in uint offset, in int64_t value, out int64_t original);
   void InterlockedMin64(in uint offset, in uint64_t value, out uint64_t original);
   void InterlockedMax64(in uint offset, in int64_t value);
   void InterlockedMax64(in uint offset, in uint64_t value);
   void InterlockedMax64(in uint offset, in int64_t value, out int64_t original);
   void InterlockedMax64(in uint offset, in uint64_t value, out uint64_t original);
\end{HLSL}

Perform the sign-approrpriate minimum or maximum comparison on
the 64-bit integer buffer value at the byte offset \texttt{offset} and the provided \texttt{value}.
Store the lesser or greater value, as appropriate, to that offset buffer location,
optionally returning the original buffer value through \texttt{original}.

\begin{HLSL}
   void InterlockedExchange64(in uint offset, in uint64_t value, out uint64_t original);
\end{HLSL}

Performs an atomic exchange of
the 64-bit integer buffer value at the byte offset \texttt{offset} with the provided \texttt{value}.
Stores \texttt{value} at the offset buffer location
and returns the original buffer value through \texttt{original}.

\begin{HLSL}
   void InterlockedCompareStore64(in uint offset, in uint64_t compare,
                                   in uint64_t value);
   void InterlockedCompareExchange64(in uint offset, in uint64_t compare,
                                      in uint64_t value, out uint64_t original);
\end{HLSL}

Perform an atomic exchange of or store to
the 64-bit integer buffer value at the byte offset \texttt{offset} with the provided \texttt{value}
if the value at that location matches the value of \texttt{compare}.
\texttt{InterlockedCompareExchange} returns the original buffer value through \texttt{original}.

\begin{HLSL}
   void InterlockedExchangeFloat(in uint byteOffest, in float value,
                                  out float original);
\end{HLSL}

Performs an atomic exchange of
the 32-bit float buffer value at the byte offset \texttt{offset} with the provided \texttt{value}.
Stores \texttt{value} at the offset buffer location
and returns the original buffer value through \texttt{original}.


\begin{HLSL}
  void InterlockedCompareStoreFloatBitwise(in uint byteOffest, in float compare,
                                            in float value);
  void InterlockedCompareExchangeFloatBitwise(in uint byteOffest,
                                               in float compare,
                                               in float value,
                                               out float original);
\end{HLSL}

Perform an atomic exchange of or store to
the 32-bit float buffer value at the byte offset \texttt{offset} with the provided \texttt{value}
if the value at that location matches the value of \texttt{compare} in a bitwise comparison.
\texttt{InterlockedCompareExchangeFloatBitwise} returns the original buffer value through \texttt{original}.

Unlike standard floating point comparisons, values will only match if the bit
representation matches which can miss some matching values that have different
bit repesantations in addition to ignoring floating point rules surrounding NaN
and infinite values.

\Sec{Structured Buffers}{Resources.stbufs}

Structured buffers are raw buffers with associated types that facilitate
indexing.
Unlike typed buffers, they perform no format conversions.
The buffer contents is treated as raw data and the casting to the given types
is done bit for bit.
Structured buffers can be defined with scalar, vector, matrix, or user-defined
struct elements.

Structured buffers can be read-only, writable, appendable, or consumable.
Writable buffers can have their elements assigned in arbitrary locations.
Append structured buffers can only have elements added to the end.
Consume structured buffers can only have elements removed from the end.

\begin{HLSL}
template <typename T>
 class StructuredBuffer {
 public:
   StructuredBuffer();
   StructuredBuffer(const StructuredBuffer &buf);
   StructuredBuffer operator=(StructuredBuffer buf);

   void GetDimensions(out uint count, out uint stride) const;

   // Element access.
   T operator[](int index) const;
   T Load(in int index) const;
   T Load(in int index, out uint status) const;
};

template <typename T>
 class RWStructuredBuffer : public StructuredBuffer {
 public:
   RWStructuredBuffer();
   RWStructuredBuffer(const RWStructuredBuffer &buf);
   RWStructuredBuffer operator=(RWStructuredBuffer buf);

   // Element assignment.
   T &operator[](int index);

   // Hidden counter increment/decrement.
   uint IncrementCounter();
   uint DecrementCounter();
};

template <typename T>
 class AppendStructuredBuffer {
 public:
   AppendStructuredBuffer();
   AppendStructuredBuffer(const AppendStructuredBuffer &buf);
   AppendStructuredBuffer operator=(AppendStructuredBuffer buf);

   void GetDimensions(out uint count, out uint stride) const;

   void Append(in T value);
};

template <typename T>
 class ConsumeStructuredBuffer {
 public:
   ConsumeStructuredBuffer();
   ConsumeStructuredBuffer(const ConsumeStructuredBuffer &buf);
   ConsumeStructuredBuffer operator=(ConsumeStructuredBuffer buf);

   void GetDimensions(out uint count, out uint stride) const;

   T Consume();
};

\end{HLSL}

\Sub{Constructors}{Resources.stbufs.ctrs}

Structured buffers must be defined with explicit template parameter types.
\begin{HLSL}
  struct S {int i; float f;};
  StructuredBuffer<S> robuf1;
  StructuredBuffer<float4> robuf2;
  StructuredBuffer<float2x3> robuf3;

  RWStructuredBuffer<S> rwbuf1;
  RWStructuredBuffer<float4> rwbuf2;
  RWStructuredBuffer<float2x3> rwbuf3;

  AppendStructuredBuffer<S> apbuf1;
  AppendStructuredBuffer<float4> apbuf2;
  AppendStructuredBuffer<float2x3> apbuf3;

  ConsumeStructuredBuffer<S> cobuf1;
  ConsumeStructuredBuffer<float4> cobuf2;
  ConsumeStructuredBuffer<float2x3> cobuf3;
\end{HLSL}

When defined at global scope, structured buffers are bound to externally-defined backing stores
using the explicit binding location if provided or the implicit binding if not (\ref{Resources.binding}).

When defined at local scope, structured buffers represent local references
that can be associated with global buffers when assigned,
but must be resolvable to a unique global buffer declaration.

\begin{note}
Resource types contained in structs are only allocated registers when they are explicitly used.
This includes elements of arrays of resources as such array elements must be indexed with literals.
\end{note}


\begin{HLSL}
  StructuredBuffer<int3x3> grobuf;
  RWStructuredBuffer<int3x3> grwbuf;
  AppendStructuredBuffer<int3x3> gapbuf;
  ConsumeStructuredBuffer<int3x3> gcobuf;
  void main() {
    StructuredBuffer<int3x3> lrobuf = grobuf;
    RWStructuredBuffer<int3x3> lrwbuf = grwbuf;
    AppendStructuredBuffer<int3x3> lbuf = gapbuf;
    ConsumeStructuredBuffer<int3x3> lcobuf = gcobuf;
  }
\end{HLSL}
Structured buffer operands to the assignment operator must have the same element type.

\Sub{Dimensions}{Resources.stbufs.dims}

The structure count and stride of a structured buffer can be retrieved using the \texttt{GetDimensions} method.
\begin{HLSL}
void GetDimensions(out uint count, out uint stride);
\end{HLSL}

This returns number of structured elements of the buffer through the \texttt{count} out parameter
and the size of each element in bytes through the \texttt{stride} out parameter.

\Sub{Element Access}{Resources.stbufs.access}

The contents of read-only StructuredBuffers and writable RWStructuredBuffers can be retrieved
using the \texttt{Load} methods or the subscript operator.

\begin{HLSL}
 T Load(in int index) const;
 T operator[](int index) const;
\end{HLSL}

These each return the element of the buffer's type at the given index \texttt{index}.

An additional \texttt{Load} method returns the indicated element just as the other,
but also takes a second \texttt{status} parameter that returns information about the accessed resource.
\begin{HLSL}
 T Load(in int index, out uint status) const;
\end{HLSL}

The \texttt{status} parameter returns an indication of whether all of the retrieved value
came from fully mapped parts of the resource.
This parameter must be passed to the built-in function \texttt{CheckAccessFullyMapped} (\ref{Resources.mapcheck})
in order to interpret it.

Writable RWStructuredBuffers have an additional subscript operator that allows assignment to a structure element of the buffer.
It behaves as if it returns a reference to that element.
\begin{HLSL}
 T &operator[](int index);
\end{HLSL}

\Sub{Counter Manipulation}{Resources.stbufs.counter}

RWStructuredBuffers may have a hidden counter that can be incremented and decremented using methods.

\begin{HLSL}
   uint IncrementCounter();
   uint DecrementCounter();
\end{HLSL}

Increment or decrements the hidden counter associated with the RWStructuredBuffer.

\Sub{Append}{Resources.stbufs.append}

AppendStructuredBuffers can only be appended to as an output stream.

\begin{HLSL}
   void Append(in T value);
\end{HLSL}

Appends the \texttt{value} of the template parameter types to the end of the AppendStructuredBuffer.
Subsequent appends will continue to add elements to the end of the buffer.

\Sub{Consume}{Resources.cnstbufs.consume}

ConsumeStructuredBuffers can only have values pulled from them as an input stream.

\begin{HLSL}
   T Consume();
\end{HLSL}

Removes and returns a value of the template parameter types from the end of the ConsumeStructuredBuffer.

\Sec{Constant Buffers}{Resources.cnbuf}

\Sec{Samplers}{Resources.samp}

\Sec{CheckAccessFullyMapped}{Resources.mapcheck}

The mapped status value returned by certain texture and buffer methods can be intrepreted by a built-in function:

\begin{HLSL}
  bool CheckAccessFullyMapped(in uint status);
\end{HLSL}

This function returns true is the value was accessed from a fully committed resource,
or from fully mapped pages of a sparse resource.
If any part of the return value was from unmapped memory, false is returned.

\Sec{Resource Binding}{Resources.binding}

Resources are bound to external memory using virtual registers within logical registers spaces.
These can be specified explicitly using attributes after the global resource declaration
or implicitly by leaving them off.

\begin{grammar}
  \define{resource-binding}\br
  \terminal{: register(} register-type bind-number \terminal{)}\br
  \terminal{: register(} register-type bind-number , \terminal{space} bind-number \terminal{)}\br
  \define{register-type} \textnormal{one of}\br
  \terminal{t u b s}\br
  \define{bind-number}\br
  digit\br
  bind-number digit\br
\end{grammar}

The register type indicates whether the binding is for a read-only (\texttt{t}), a writable (\texttt{u}),
a constant buffer (\texttt{b}), or a sampler (\texttt{s}).
The register bind number indicates the register number at which the resource variable begins.
The optional second parameter indicates the virtual register space that the register belongs to.

Sample syntax of virtual register binding attributes:
\begin{HLSL}
  Buffer<float> robuf : register(t0, space0);
  RWBuffer<float> rwbuf : register(u0, space0);
  SamplerState samp : register(s0, space0);
  cbuffer cbuf : register(b0, space0) { float f; };
\end{HLSL}

Each resource must have a unique register. Different resources cannot occupy the same register,
but read-only and writable buffers have separate namespaces and different logical register spaces have independent
sets of register numbers. Meaning the following is all allowed:
\begin{HLSL}
  Buffer<float> mybuf : register(t0, space0);
  Buffer<float> yourbuf : register(t0, space1);
  RWBuffer<float> hisbuf : register(u0, space0);
  RWBuffer<float> herbuf : register(u0, space1);
\end{HLSL}

An aggregate resource type can be:
\begin{itemize}
\item a struct containing one or more resources types
\item an array of resource types
\item an array or struct containing either of the above
\end{itemize}

Aggregate resource types with register annotations
occupy the first register they specify and however many additional sequential registers
the aggregate requires to assign a register to each of its elements that correspond to the given register type.
Multiple register annotations can be placed on aggregates involving structs where each corresponds to a distinct
register type that will apply to all the registers of that type in the aggregate.

\begin{note}
Resource types contained in structs are only allocated registers when they are explicitly used.
This includes elements of arrays of resources since such array elements must be indexed with literals.
\end{note}

\begin{HLSL}
  // Occupies t0, t1, and t2.
  Buffer<float> fbuf[3] : register(t0, space0);
  // Occupies registers t3 - t14.
  Buffer<int4> ibuf[4][3] : register(t3, space0);
  // Occupies registers t15 and u0.
  struct {RWBuffer<int> rwbuf; Buffer buf;} sbufs : register(t15) : register(u0);
  // Occupies registers t16 - t21.
  struct {Buffer bufs[5]; Buffer buf;} robufs : register(t16);
  // Occupies registers t22-24 and u1-u2
  struct {Buffer r; RWBuffer<int> w;} bibufs[2] : register(t22) : register(u1);
\end{HLSL}

If the register binding or register space is not specified, implicit values are used.
Whenever not specified, the space defaults to \texttt{space0}.

\begin{HLSL}
  Buffer<float2> buf1 : register(t0, space0);
  Buffer<float2> buf2 : register(t1); // defaults to space0
\end{HLSL}

When the register is not specified, resources will recieve implementation-dependent register assignments.
