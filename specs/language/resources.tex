\Ch{Resource Types}{Resources}

Resources are built-in intangible types representing memory with an external interface.

These take the form of Typed Buffers, Raw Buffers, Textures, and Samplers.
Buffer and Texture types can be read-only or writable.

\Sec{Typed Buffers}{Resources.tybuf}

The typed buffer class template represents a one-dimensional resource containing an array of a single given type.
Its contents are indexed by typed access to each element
Types may have their formats converted upon load.

Template types can be any type that totals or can be padded to 16 bytes.

All typed buffers can be read through subscript operators or Load methods.
Writable typed buffers can be written through subscript operators.

Where \texttt{T} can be any arithmetic type \ref{Basic.types.arithmetic}
or a vector with a maximum of four elements containing such an arithmetic type.
The total size of a single contained element must be less than 16 bytes.

Typed buffers perform format conversions on load such that the underlying data
gets converted to the destination type.

\Sec{Read-only Typed Buffers}{Resources.rotybufs}

\begin{HLSL}
template <typename T = float4>
 class Buffer {
   Buffer();
   Buffer operator=(Buffer buf);

   void GetDimensions(out uint width);

   // element access
   T operator[](uint pos);
   T Load(in int pos);
   T Load(in int pos, out uint status);
};
\end{HLSL}

\Sub{Constructors}{Resources.rotybufs.ctrs}

Buffers can be defined with explicit or implicit template parameter types.
\begin{HLSL}
  Buffer buf1;
  Buffer<> buf2;
  Buffer<float4> buf3;
\end{HLSL}
Since \texttt{float4} is the default type, all of these definitions are equivalent.

When defined at global scope, Buffers are bound to externally-defined backing stores
using the explicit binding location if provided or the implicit binding if not.

When defined at local scope, Buffers represent local references
that can be associated with global buffers when assigned,
but must be resolvable to a unique global buffer.

\begin{HLSL}
  Buffer<int> gbuf;
  void doit() {
    Buffer lbuf<int> = gbuf;
  }
\end{HLSL}
Buffer operands to the assignment operator must have the same element type.

\Sub{Dimensions}{Resources.rotybufs.dims}

The size of a Buffer can be retrieved using the \texttt{GetDimensions} method.
\begin{HLSL}
void GetDimensions(out uint size);
\end{HLSL}

This returns the full size of the buffer in bytes through the \texttt{size} out parameter.

\Sub{Element Access}{Resources.rotybufs.access}

The contents of Buffers can be retrieved using the \texttt{Load} methods
or the subscript operator.

\begin{HLSL}
 T Load(in int pos);
 T operator[](uint pos);
\end{HLSL}

These each return the element at the given position \texttt{pos} of the type specified in the buffer definition.

An additional \texttt{Load} method returns the indicated element just as the other,
but also takes a second \texttt{status} parameter that returns information about the accessed resource.
\begin{HLSL}
 T Load(in int pos, out uint status);
\end{HLSL}

The \texttt{status} parameter returns an indication of whether the value retrieved accessed a mapped tile
resource. This parameter is interpretted using the \texttt{CheckAccessFullyMapped}
built-in function. If tiled resources are enabled, it returns true if the value was accessed from a mapped
tile resource and false otherwise. If tiled resourced are not enabled, it will return true.

\Sec{Writable Typed Buffers}{Resources.rwtybufs}

\begin{HLSL}
template <typename T>
 class RWBuffer {

   RWBuffer();
   RWBuffer operator=(RWBuffer buf);

   void GetDimensions(out uint size);

   // element access
   T operator[](uint n);
   T Load(in int pos);
   T Load(in int pos, out uint status);
};
\end{HLSL}

\Sub{Constructors}{Resources.rwtybufs.ctrs}

RWBuffers must be defined with explicit template parameter types.
\begin{HLSL}
  RWBuffer<float4> buf;
\end{HLSL}

When defined at global scope, RWBuffers are bound to externally-defined backing stores
using the explicit binding location if provided or the implicit binding if not.

When defined at local scope, RWBuffers represent local references
that can be associated with global buffers when assigned,
but must be resolvable to a unique global buffer.

\begin{HLSL}
  RWBuffer<int> gbuf;
  void doit() {
    RWBuffer lbuf<int> = gbuf;
  }
\end{HLSL}
RWBuffer operands to the assignment operator must have the same element type.

\Sub{Dimensions}{Resources.rwtybufs.dims}

The size of a RWBuffer can be retrieved using the \texttt{GetDimensions} method.
\begin{HLSL}
void GetDimensions(out uint size);
\end{HLSL}

This returns the full size of the buffer in bytes through the \texttt{width} out parameter.

\Sub{Element Access}{Resources.rwtybufs.access}

The contents of RWBuffers can be retrieved using the \texttt{Load} methods
or the subscript operator.

\begin{HLSL}
 T Load(in int pos);
 T operator[](uint pos);
\end{HLSL}

These each return the element at the given position \texttt{pos} of the type specified in the buffer definition.

An additional \texttt{Load} method returns the indicated element just as the other,
but also takes a second \texttt{status} parameter that returns information about the accessed resource.
\begin{HLSL}
 T Load(in int pos, out uint status);
\end{HLSL}

The \texttt{status} parameter returns an indication of whether the value retrieved accessed a mapped tile
resource. This parameter is interpretted using the \texttt{CheckAccessFullyMapped}
built-in function. If tiled resources are enabled, it returns true if the value was accessed from a mapped
tile resource and false otherwise. If tiled resourced are not enabled, it will return true.

Writable buffers have an additional subscript operator that allows assignment to an element of the buffer.
It behaves as if it returns a reference to that element.
\begin{HLSL}
 T &operator[](uint pos);
\end{HLSL}

\Sec{Raw Buffer Types}{Resources.rawbufs}

Raw buffers are one-dimensional resources of arbitrary memory layout.
They are either ByteAddressBuffers or StructuredBuffers.
ByteAddressBuffers enable per-byte access to the raw data while
StructuredBuffers have an associated structure type that determines how they are
indexed. This type can be a scalar, vector, matrix, or user-defined struct.

Partial writes aren't allowed.
Though structured buffers can contain vectors,
assignment to individual vector elements will result in an error.

\Sec{Read-only Byte Access Buffers}{Resources.robabufs}

\begin{HLSL}
 class ByteAddressBuffer {

   ByteAddressBuffer();
   ByteAddressBuffer operator=(ByteAddressBuffer buf);

   void GetDimensions(out uint size);

   // element access
   uint Load(in uint byteOffset)
   uint2 Load2(in uint byteOffset)
   uint3 Load3(in uint byteOffset)
   uint4 Load4(in uint byteOffset)
   template<typename T>
   T Load(in uint byteOffset);

   uint Load(in uint byteOffset, out uint status)
   uint2 Load2(in uint byteOffset, out uint status)
   uint3 Load3(in uint byteOffset, out uint status)
   uint4 Load4(in uint byteOffset, out uint status)
   template<typename T>
   T Load(in uint byteOffset, out uint status);
};
\end{HLSL}

\Sub{Constructors}{Resources.robabufs.ctrs}

Since their contents are viewed as raw bytes, ByteAddressBuffers are defined without any type parameters.
\begin{HLSL}
  ByteAddressBuffer buf;
\end{HLSL}

When defined at global scope, ByteAccessBuffers are bound to externally-defined backing stores
using the explicit binding location if provided or the implicit binding if not.

When defined at local scope, ByteAccessBuffers represent local references
that can be associated with global ByteAccessBuffers when assigned,
but must be resolvable to a unique global buffer.

\begin{HLSL}
  ByteAddressBuffer gbuf;
  void doit() {
    ByteAddressBuffer lbuf = gbuf;
  }
\end{HLSL}

\Sub{Dimensions}{Resources.robabufs.dims}

The size of a ByteAddressBuffer can be retrieved using the \texttt{GetDimensions} method.
\begin{HLSL}
void GetDimensions(out uint size);
\end{HLSL}

This returns the full size of the buffer in bytes through the \texttt{width} out parameter.

\Sub{Element Access}{Resources.robabufs.access}

The contents of ByteAddressBuffers can be retrieved using the \texttt{Load} methods.

\begin{HLSL}
 uint Load(in uint byteOffset)
 uint2 Load2(in uint byteOffset)
 uint3 Load3(in uint byteOffset)
 uint4 Load4(in uint byteOffset)
\end{HLSL}

These each return the bytes at the given position \texttt{byteOffset} in the form of one or more unsigned integers.
The \texttt{byteOffset} address is in bytes and must be a multiple of 4.
Each of the variants returns a number of bytes corresponding to the size of the uint vector returned.

An additional templated load method allows returning the bytes at the given byte offset in the form
of the type given in the template parameter. This can be a scalar, vector, matrix, or user-defined struct.

\begin{HLSL}
  template<typename T>
  T Load(in uint byteOffset);
\end{HLSL}

The alignment requirements of \texttt{byteOffset} for this operation is the size in bytes of the largest
scalar type contained in the given tyep \texttt{T}.

Additional \texttt{Load} methods return the same values as the others,
but also take a second \texttt{status} parameter that returns information about the accessed resource.
\begin{HLSL}
  uint Load(in uint byteOffset, out uint status)
  uint2 Load2(in uint byteOffset, out uint status)
  uint3 Load3(in uint byteOffset, out uint status)
  uint4 Load4(in uint byteOffset, out uint status)
  template<typename T>
  T Load(in uint byteOffset, out uint status);
\end{HLSL}

The \texttt{status} parameter returns an indication of whether the value retrieved accessed a mapped tile
resource. This parameter is interpretted using the \texttt{CheckAccessFullyMapped}
built-in function. If tiled resources are enabled, it returns true if the value was accessed from a mapped
tile resource and false otherwise. If tiled resourced are not enabled, it will return true.

\Sec{Writable Byte Access Buffers}{Resources.rwbabufs}

\begin{HLSL}
 class RWByteAddressBuffer {

   RWByteAddressBuffer();
   RWByteAddressBuffer operator=(RWByteAddressBuffer buf);

   void GetDimensions(out uint size);

   // element access
   uint Load(in uint byteOffset)
   uint2 Load2(in uint byteOffset)
   uint3 Load3(in uint byteOffset)
   uint4 Load4(in uint byteOffset)
   template<typename T>
   T Load(in uint byteOffset);

   uint Load(in uint byteOffset, out uint status)
   uint2 Load2(in uint byteOffset, out uint status)
   uint3 Load3(in uint byteOffset, out uint status)
   uint4 Load4(in uint byteOffset, out uint status)
   template<typename T>
   T Load(in uint byteOffset, out uint status);

   void Store(in uint byteOffset, in uint value)
   void Store2(in uint byteOffset, in uint2 value)
   void Store3(in uint byteOffset, in uint3 value)
   void Store4(in uint byteOffset, in uint4 value)

   template<typename T>
   void Store(in uint byteOffset, in T value)

   // 32-bit integer atomic arithmetic/bitwise operations
   void InterlockedAdd(in uint pos, in uint value);
   void InterlockedAdd(in uint pos, in uint value, out uint original)
   void InterlockedAnd(in uint pos, in uint value);
   void InterlockedAnd(in uint pos, in uint value, out uint original)
   void InterlockedOr(in uint pos, in uint value);
   void InterlockedOr(in uint pos, in uint value, out uint original)
   void InterlockedXor(in uint pos, in uint value);
   void InterlockedXor(in uint pos, in uint value, out uint original)

   // 32-bit integer atomic comparison operations
   void InterlockedMin(in uint pos, in int value)
   void InterlockedMin(in uint pos, in uint value)
   void InterlockedMin(in uint pos, in int value, out int original)
   void InterlockedMin(in uint pos, in uint value, out uint original)
   void InterlockedMax(in uint pos, in int value)
   void InterlockedMax(in uint pos, in uint value)
   void InterlockedMax(in uint pos, in int value, out int original)
   void InterlockedMax(in uint pos, in uint value, out uint original)

   // 32-bit integer atomic compare/exchange operations
   void InterlockedCompareStore(in uint pos, in uint compare, in uint value);
   void InterlockedExchange(in uint pos, in uint value, out uint original);
   void InterlockedCompareExchange(in uint pos, in uint compare, in uint value,
                                    out uint original);

   // 64-bit integer atomic arithmatic/bitwise operations
   void InterlockedAdd64(in uint pos, in uint64_t value);
   void InterlockedAdd64(in uint pos, in uint64_t value, out uint64_t original)
   void InterlockedAnd64(in uint pos, in uint64_t value);
   void InterlockedAnd64(in uint pos, in uint64_t value, out uint64_t original)
   void InterlockedOr64(in uint pos, in uint64_t value);
   void InterlockedOr64(in uint pos, in uint64_t value, out uint64_t original)
   void InterlockedXor64(in uint pos, in uint64_t value);
   void InterlockedXor64(in uint pos, in uint64_t value, out uint64_t original)

   // 64-bit integer atomic comparison operations
   void InterlockedMin64(in uint pos, in int64_t value)
   void InterlockedMin64(in uint pos, in int64_t value, out int64_t original)
   void InterlockedMin64(in uint pos, in uint64_t value)
   void InterlockedMin64(in uint pos, in uint64_t value, out uint64_t original)
   void InterlockedMax64(in uint pos, in int64_t value)
   void InterlockedMax64(in uint pos, in int64_t value, out int64_t original)
   void InterlockedMax64(in uint pos, in uint64_t value)
   void InterlockedMax64(in uint pos, in uint64_t value, out uint64_t original)

   // 64-bit integer atomic compare/exchange operations
   void InterlockedCompareStore64(in uint pos, in uint64_t compare, in uint64_t value);
   void InterlockedExchange64(in uint pos, in int64_t value, out int64_t original);
   void InterlockedCompareExchange64(in uint pos, in uint64_t compare,
                                      in uint64_t value, out int64_t original);

   // 32-bit float atomic compare/exchange operations
   void InterlockedCompareStoreFloatBitwise(in uint byteOffest,
                                             in float compare, in float value);
   void InterlockedExchangeFloat(in uint byteOffest, in float value,
                                  out float original);
   void InterlockedCompareExchangeFloatBitwise(in uint byteOffest,
                                                in float compare,
                                                in float value,
                                                out float original);
};
\end{HLSL}

\Sub{Constructors}{Resources.rwbabufs.ctrs}

Since their contents are viewed as raw bytes, RWByteAddressBuffers are defined as a simple variable definition.
\begin{HLSL}
  RWByteAddressBuffer buf;
\end{HLSL}

When defined at global scope, RWByteAccessBuffers are bound to externally-defined backing stores
using the explicit binding location if provided or the implicit binding if not.

When defined at local scope, RWByteAccessBuffers represent local references
that can be associated with global RWByteAccessBuffers when assigned,
but must be resolvable to a unique global buffer.

\begin{HLSL}
  RWByteAddressBuffer gbuf;
  void doit() {
    RWByteAddressBuffer lbuf = gbuf;
  }
\end{HLSL}

\Sub{Dimensions}{Resources.rwbabufs.dims}

The size of a RWByteAddressBuffer can be retrieved using the \texttt{GetDimensions} method.
\begin{HLSL}
void GetDimensions(out uint size);
\end{HLSL}

This returns the full size of the buffer in bytes through the \texttt{width} out parameter.

\Sub{Element Access}{Resources.rwbabufs.access}

The contents of RWByteAccessBuffers can be retrieved using the \texttt{Load} methods.

\begin{HLSL}
 uint Load(in uint byteOffset)
 uint2 Load2(in uint byteOffset)
 uint3 Load3(in uint byteOffset)
 uint4 Load4(in uint byteOffset)
\end{HLSL}

These each return the bytes at the given position \texttt{byteOffset} in the form of one or more unsigned integers.
The \texttt{byteOffset} address is in bytes and must be a multiple of 4.
Each of the variants returns a number of bytes corresponding to the size of the uint vector returned.

An additional templated load method allows returning the bytes at the given byte offset in the form
of the type given in the template parameter. This can be a scalar, vector, matrix, or user-defined struct.

\begin{HLSL}
  template<typename T>
  T Load(in uint byteOffset);
\end{HLSL}

The alignment requirements of \texttt{byteOffset} for this operation is the size in bytes of the largest
scalar type contained in the given tyep \texttt{T}.

Additional \texttt{Load} methods return the same values as the others,
but also take a second \texttt{status} parameter that returns information about the accessed resource.
\begin{HLSL}
  uint Load(in uint byteOffset, out uint status)
  uint2 Load2(in uint byteOffset, out uint status)
  uint3 Load3(in uint byteOffset, out uint status)
  uint4 Load4(in uint byteOffset, out uint status)
  template<typename T>
  T Load(in uint byteOffset, out uint status);
\end{HLSL}

The \texttt{status} parameter returns an indication of whether the value retrieved accessed a mapped tile
resource. This parameter is interpretted using the \texttt{CheckAccessFullyMapped}
built-in function. If tiled resources are enabled, it returns true if the value was accessed from a mapped
tile resource and false otherwise. If tiled resourced are not enabled, it will return true.


\Sub{Atomic Operations}{Resources.rwbabufs.atomics}

RWByteAddressBuffers have a suite of atomic methods that perform the given operation
on signed or unsigned integer element data indexed by the given integer size
in a way that ensures no contention from other threads performing other operations.
Each has an overload that optionally returns the original value before the atomic operation was performed.

\begin{HLSL}
   void InterlockedAdd(in uint pos, in uint value);
   void InterlockedAdd(in uint pos, in uint value, out uint original)
   void InterlockedAnd(in uint pos, in uint value);
   void InterlockedAnd(in uint pos, in uint value, out uint original)
   void InterlockedOr(in uint pos, in uint value);
   void InterlockedOr(in uint pos, in uint value, out uint original)
   void InterlockedXor(in uint pos, in uint value);
   void InterlockedXor(in uint pos, in uint value, out uint original)
\end{HLSL}

Performs atomic addition, bitwise and, bitwise or, or bitwise exclusive or
on the element found in 32-bit integer position \texttt{pos} and the provided \texttt{value}
and stores the resulting sum in that buffer position,
optionally returning the orignal value of the buffer position through \texttt{original}

\begin{HLSL}
   void InterlockedMin(in uint pos, in int value)
   void InterlockedMin(in uint pos, in uint value)
   void InterlockedMin(in uint pos, in int value, out int original)
   void InterlockedMin(in uint pos, in uint value, out uint original)
   void InterlockedMax(in uint pos, in int value)
   void InterlockedMax(in uint pos, in uint value)
   void InterlockedMax(in uint pos, in int value, out int original)
   void InterlockedMax(in uint pos, in uint value, out uint original)
\end{HLSL}

Performs the sign-approrpriate minimum or maximum comparison of the 32-bit integer indexed by \texttt{pos}
and the provided \texttt{value} and assigns the less or greater as appropriate to that buffer position,
optionally returning the orignal value at that buffer position through \texttt{original}

\begin{HLSL}
   void InterlockedExchange(in uint pos, in uint value, out uint original);
\end{HLSL}

Performs an atomic exchange of the the 32-bit integer indexed by \texttt{pos} with
the value of \texttt{value} and returns the original value at that position through \texttt{original}.

\begin{HLSL}
   void InterlockedCompareStore(in uint pos, in uint compare, in uint value);
   void InterlockedCompareExchange(in uint pos, in uint compare, in uint value,
                                    out uint original);
\end{HLSL}

These perform an atomic exchange of or store to the the 32-bit integer indexed by \texttt{pos}
with the provided \texttt{value} if the value at that position matches the value of \texttt{compare}.
\texttt{InterlockedCompareExchange} returns the original value at that position through \texttt{original}.

\begin{HLSL}
   void InterlockedAdd64(in uint pos, in uint64_t value);
   void InterlockedAdd64(in uint pos, in uint64_t value, out uint64_t original)
   void InterlockedAnd64(in uint pos, in uint64_t value);
   void InterlockedAnd64(in uint pos, in uint64_t value, out uint64_t original)
   void InterlockedOr64(in uint pos, in uint64_t value);
   void InterlockedOr64(in uint pos, in uint64_t value, out uint64_t original)
   void InterlockedXor64(in uint pos, in uint64_t value);
   void InterlockedXor64(in uint pos, in uint64_t value, out uint64_t original)
\end{HLSL}

Performs atomic addition, bitwise and, bitwise or, or bitwise exclusive or
on the element found in 64-bit integer position \texttt{pos} and the provided \texttt{value}
and stores the resulting sum in that buffer position,
optionally returning the orignal value of the buffer position through \texttt{original}

\begin{HLSL}
   void InterlockedMin64(in uint pos, in int64_t value)
   void InterlockedMin64(in uint pos, in uint64_t value)
   void InterlockedMin64(in uint pos, in int64_t value, out int64_t original)
   void InterlockedMin64(in uint pos, in uint64_t value, out uint64_t original)
   void InterlockedMax64(in uint pos, in int64_t value)
   void InterlockedMax64(in uint pos, in uint64_t value)
   void InterlockedMax64(in uint pos, in int64_t value, out int64_t original)
   void InterlockedMax64(in uint pos, in uint64_t value, out uint64_t original)
\end{HLSL}

Performs the sign-approrpriate minimum or maximum comparison of the 64-bit integer indexed by \texttt{pos}
and the provided \texttt{value} and assigns the less or greater as appropriate to that buffer position,
optionally returning the orignal value at that buffer position through \texttt{original}

\begin{HLSL}
   void InterlockedExchange64(in uint pos, in uint64_t value, out uint64_t original);
\end{HLSL}

Performs an atomic exchange of the the 64-bit integer indexed by \texttt{pos} with
the value of \texttt{value} and returns the original value at that position through \texttt{original}.

\begin{HLSL}
   void InterlockedCompareStore64(in uint pos, in uint64_t compare,
                                   in uint64_t value);
   void InterlockedCompareExchange64(in uint pos, in uint64_t compare,
                                      in uint64_t value, out uint64_t original);
\end{HLSL}

These perform an atomic exchange of or store to the the 64-bit integer indexed by \texttt{pos}
with the provided \texttt{value} if the value at that position matches the value of \texttt{compare}.
\texttt{InterlockedCompareExchange64} returns the original value at that position through \texttt{original}.

\begin{HLSL}
   void InterlockedExchangeFloat(in uint byteOffest, in float value,
                                  out float original);
\end{HLSL}

Performs an atomic exchange of the the 32-bit float indexed by \texttt{pos} with
the value of \texttt{value} and returns the original value at that position through \texttt{original}.

\begin{HLSL}
  void InterlockedCompareStoreFloatBitwise(in uint byteOffest, in float compare,
                                            in float value);
  void InterlockedCompareExchangeFloatBitwise(in uint byteOffest,
                                               in float compare,
                                               in float value,
                                               out float original);
\end{HLSL}

These perform an atomic exchange of or store to the the 32-bit float indexed by \texttt{pos}
with the provided \texttt{value} if the value at that position matches the value of \texttt{compare}
in a bitwise comparison.
Unlike standard floating point comparisons, values will only match if the bit representation matches
which can miss some matching values that have different bit repesantations in addition to ignoring
floating point rules surrounding NaN and infinite values.
\texttt{InterlockedCompareExchangeFloatBitwise} returns the original value at that position through \texttt{original}.

\Sec{Structured Buffers}{Resources.stbufs}

StructuredBuffers are raw buffers with associated types that facilitate indexing.
Unlike typed buffers, they perform no format conversions.
The buffer contents is treated as raw data and the casting to the given types is done bit for bit.
StructuredBuffers can be defined with scalar, vector, matrix, or user-defined struct elements.

\Sec{Read-only Structured Buffers}{Resources.rostbufs}

\begin{HLSL}
template <typename T>
 class StructuredBuffer {
   StructuredBuffer();
   StructuredBuffer operator=(StructuredBuffer buf);

   void GetDimensions(out uint count, out uint stride);

   // element access
   T operator[](uint pos);
   T Load(in int pos);
   T Load(in int pos, out uint status);
};
\end{HLSL}

\Sub{Constructors}{Resources.rostbufs.ctrs}

StructuredBuffers must be defined with explicit template parameter types.
\begin{HLSL}
  struct S {int i; float f;};
  StructuredBuffer<S> buf1;
  StructuredBuffer<float4> buf2;
  StructuredBuffer<float2x3> buf3;
\end{HLSL}

When defined at global scope, StructuredBuffers are bound to externally-defined backing stores
using the explicit binding location if provided or the implicit binding if not.

When defined at local scope, StructuredBuffers represent local references
that can be associated with global buffers when assigned,
but must be resolvable to a unique global buffer.

\begin{HLSL}
  StructuredBuffer<int3x3> gbuf;
  void doit() {
    StructuredBuffer lbuf<int3x3> = gbuf;
  }
\end{HLSL}
StructuredBuffer operands to the assignment operator must have the same element type.

\Sub{Dimensions}{Resources.rostbufs.dims}

The structure count and stride of a StructuredBuffer can be retrieved using the \texttt{GetDimensions} method.
\begin{HLSL}
void GetDimensions(out uint count, out uint stride);
\end{HLSL}

This returns number of structured elements of the buffer through the \texttt{count} out parameter
and the size of each element in bytes through the \texttt{stride} out parameter.

\Sub{Element Access}{Resources.rostbufs.access}

The contents of StructuredBuffers can be retrieved using the \texttt{Load} methods
or the subscript operator.

\begin{HLSL}
 T Load(in int pos);
 T operator[](uint pos);
\end{HLSL}

These each return the element at the given position \texttt{pos} of the type specified in the buffer definition.

An additional \texttt{Load} method returns the indicated element just as the other,
but also takes a second \texttt{status} parameter that returns information about the accessed resource.
\begin{HLSL}
 T Load(in int pos, out uint status);
\end{HLSL}

The \texttt{status} parameter returns an indication of whether the value retrieved accessed a mapped tile
resource. This parameter is interpretted using the \texttt{CheckAccessFullyMapped}
built-in function. If tiled resources are enabled, it returns true if the value was accessed from a mapped
tile resource and false otherwise. If tiled resourced are not enabled, it will return true.

\Sec{Writable Structured Buffers}{Resources.rwstbufs}

\begin{HLSL}
template <typename T>
 class RWStructuredBuffer {
   RWStructuredBuffer();
   RWStructuredBuffer operator=(RWStructuredBuffer buf);

   void GetDimensions(out uint count, out uint stride);

   // element access
   T operator[](uint pos);
   T Load(in int pos);
   T Load(in int pos, out uint status);

   // Hidden counter increment/decrement
   uint IncrementCounter();
   uint DecrementCounter();

};
\end{HLSL}

\Sub{Constructors}{Resources.rwstbufs.ctrs}

RWStructuredBuffers must be defined with explicit template parameter types.
\begin{HLSL}
  struct S {int i; float f;};
  RWStructuredBuffer<S> buf1;
  RWStructuredBuffer<float4> buf2;
  RWStructuredBuffer<float2x3> buf3;
\end{HLSL}

When defined at global scope, RWStructuredBuffers are bound to externally-defined backing stores
using the explicit binding location if provided or the implicit binding if not.

When defined at local scope, RWStructuredBuffers represent local references
that can be associated with global buffers when assigned,
but must be resolvable to a unique global buffer.

\begin{HLSL}
  RWStructuredBuffer<int3x3> gbuf;
  void doit() {
    RWStructuredBuffer lbuf<int3x3> = gbuf;
  }
\end{HLSL}
RWStructuredBuffer operands to the assignment operator must have the same element type.

\Sub{Dimensions}{Resources.rwstbufs.dims}

The structure count and stride of a RWStructuredBuffer can be retrieved using the \texttt{GetDimensions} method.
\begin{HLSL}
void GetDimensions(out uint count, out uint stride);
\end{HLSL}

This returns number of structured elements of the buffer through the \texttt{count} out parameter
and the size of each element in bytes through the \texttt{stride} out parameter.

\Sub{Element Access}{Resources.rwstbufs.access}

The contents of RWStructuredBuffers can be retrieved using the \texttt{Load} methods
or the subscript operator.

\begin{HLSL}
 T Load(in int pos);
 T operator[](uint pos);
\end{HLSL}

These each return the element at the given position \texttt{pos} of the type specified in the buffer definition.

An additional \texttt{Load} method returns the indicated element just as the other,
but also takes a second \texttt{status} parameter that returns information about the accessed resource.
\begin{HLSL}
 T Load(in int pos, out uint status);
\end{HLSL}

The \texttt{status} parameter returns an indication of whether the value retrieved accessed a mapped tile
resource. This parameter is interpretted using the \texttt{CheckAccessFullyMapped}
built-in function. If tiled resources are enabled, it returns true if the value was accessed from a mapped
tile resource and false otherwise. If tiled resourced are not enabled, it will return true.

\Sub{Counter Manipulation}{Resources.rwstbufs.counter}

\begin{HLSL}
   uint IncrementCounter();
   uint DecrementCounter();
\end{HLSL}

Increments or decrements the hidden counter associated with the RWStructuredbuffer.

\Sec{AppendStructuredBuffer}{Resources.apstbufs}

AppendStructuredBuffers are buffers that can only be appended to as an output stream.

\begin{HLSL}
template <typename T>
 class AppendStructuredBuffer {
   AppendStructuredBuffer();
   AppendStructuredBuffer operator=(AppendStructuredBuffer buf);

   void GetDimensions(out uint count, out uint stride);

   void Append(in T value);
};
\end{HLSL}

\Sub{Constructors}{Resources.apstbufs.ctrs}

AppendStructuredBuffers must be defined with explicit template parameter types.
\begin{HLSL}
  struct S {int i; float f;};
  AppendStructuredBuffer<S> buf1;
  AppendStructuredBuffer<float4> buf2;
  AppendStructuredBuffer<float2x3> buf3;
\end{HLSL}

When defined at global scope, AppendStructuredBuffers are bound to externally-defined backing stores
using the explicit binding location if provided or the implicit binding if not.

When defined at local scope, AppendStructuredBuffers represent local references
that can be associated with global buffers when assigned,
but must be resolvable to a unique global buffer.

\begin{HLSL}
  AppendStructuredBuffer<int3x3> gbuf;
  void doit() {
    AppendStructuredBuffer lbuf<int3x3> = gbuf;
  }
\end{HLSL}
AppendStructuredBuffer operands to the assignment operator must have the same element type.

\Sub{Dimensions}{Resources.apstbufs.dims}

The structure count and stride of a AppendStructuredBuffer can be retrieved using the \texttt{GetDimensions} method.
\begin{HLSL}
void GetDimensions(out uint count, out uint stride);
\end{HLSL}

This returns number of structured elements of the buffer through the \texttt{count} out parameter
and the size of each element in bytes through the \texttt{stride} out parameter.

\Sub{Append}{Resources.apstbufs.append}

\begin{HLSL}
   void Append(in T value )
\end{HLSL}

Appends the \texttt{value} of the template parameter types to the end of the AppendStructuredBuffer.
Subsequent appends will continue to add elements to the end of the buffer.

\Sec{ConsumeStructuredBuffer}{Resources.cnstbufs}

ConsumeStructuredBuffers are buffers that can only have values pulled from them as an input stream.

\begin{HLSL}
template <typename T>
 class ConsumeStructuredBuffer {
   ConsumeStructuredBuffer();
   ConsumeStructuredBuffer operator=(ConsumeStructuredBuffer buf);

   void GetDimensions(out uint count, out uint stride);

   void Consume();
};
\end{HLSL}

\Sub{Constructors}{Resources.cnstbufs.ctrs}

ConsumeStructuredBuffers must be defined with explicit template parameter types.
\begin{HLSL}
  struct S {int i; float f;};
  ConsumeStructuredBuffer<S> buf1;
  ConsumeStructuredBuffer<float4> buf2;
  ConsumeStructuredBuffer<float2x3> buf3;
\end{HLSL}

When defined at global scope, ConsumeStructuredBuffers are bound to externally-defined backing stores
using the explicit binding location if provided or the implicit binding if not.

When defined at local scope, ConsumeStructuredBuffers represent local references
that can be associated with global buffers when assigned,
but must be resolvable to a unique global buffer.

\begin{HLSL}
  ConsumeStructuredBuffer<int3x3> gbuf;
  void doit() {
    ConsumeStructuredBuffer lbuf<int3x3> = gbuf;
  }
\end{HLSL}
ConsumeStructuredBuffer operands to the assignment operator must have the same element type.

\Sub{Dimensions}{Resources.cnstbufs.dims}

The structure count and stride of a ConsumeStructuredBuffer can be retrieved using the \texttt{GetDimensions} method.
\begin{HLSL}
void GetDimensions(out uint count, out uint stride);
\end{HLSL}

This returns number of structured elements of the buffer through the \texttt{count} out parameter
and the size of each element in bytes through the \texttt{stride} out parameter.

\Sub{Consume}{Resources.cnstbufs.consume}

\begin{HLSL}
   void Consume();
\end{HLSL}

Removes and returns a value of the template parameter types from the end of the ConsumeStructuredBuffer.

\Sec{Constant Buffers}{Resources.cnbuf}

\Sec{Samplers}{Resources.samp}


\Sec{Resource Binding}{Resources.binding}

Resources are bound to external memory using virtual registers within logical registers spaces.
These can be specified explicitly using attributes after the global resource declaration
or implicitly by leaving them off.

\begin{grammar}
  \define{resource-binding}\br
  \terminal{: register(} register-type bind-number \terminal{)}\br
  \terminal{: register(} register-type bind-number , \terminal{space} bind-number \terminal{)}\br

  \define{register-type} \textnormal{one of}\br
  \terminal{t u b s}\br

  \define{bind-number}\br
  digit\br
  bind-number digit\br
\end{grammar}

The register type indicates whether the binding is for a read-only (\texttt{t}), a writable (\texttt{u}),
a constant buffer (\texttt{b}), or a sampler (\texttt{s}).
The register bind number indicates the register number at which the resource variable begins.
The optional second parameter indicates the virtual register space that the register belongs to.

Sample syntax of virtual register binding attributes:
\begin{HLSL}
  Buffer<float> robuf : register(t0, space0);
  RWBuffer<float> rwbuf : register(u0, space0);
  SamplerState samp : register(s0, space0);
  cbuffer cbuf : register(b0, space0) { float f; };
\end{HLSL}

Each resource must have a unique register. Different resources cannot occupy the same register,
but read-only and writable buffers have separate namespaces and different logical register spaces have independent
sets of register numbers. Meaning the following is all allowed:
\begin{HLSL}
  Buffer<float> mybuf : register(t0, space0);
  Buffer<float> yourbuf : register(t0, space1);
  RWBuffer<float> hisbuf : register(u0, space0);
  RWBuffer<float> herbuf : register(u0, space1);
\end{HLSL}

Resource aggregates structs containing multiple resources types or arrays of resource types or structs containing one or more resource types.
When aggregate types have register annotations, they occupy the first register they specify as well as however many additional sequential registers
the aggregate requires to assign a register to each of its elements that correspond to the given register type.
Multiple register annotations can be placed on aggregates involving structs where each corresponds to a distinct
register type that will apply to all the registers of that type in the aggregate.

\begin{note}
Resource types contained in structs are only allocated registers when they are explicitly used.
This includes elements of arrays of resources as such array elements must be indexed with literals.
\end{note}

\begin{HLSL}
  Buffer<float> fbuf[3] : register(t0, space0); // Occupies t0, t1, and t2.
  Buffer<int4> ibuf[4][3] : register(t3, space0); // Occupies registers t3 - t14.
  struct {RWBuffer<int> rwbuf; Buffer buf;} sbufs : register(t15) : register(u0); // Occupies registers t15 and u0.
  struct {Buffer bufs[5]; Buffer buf;} robufs : register(t16); // Occupies registers t16 - t21.
  struct {Buffer r; RWBuffer<int> w;} bibufs[2] : register(t22) : register(u1); // Occupies registers t22-24 and u1-u2
\end{HLSL}

If the register binding or register space is not specified, implicit values are used.
Whenever not specified, the space defaults to \texttt{space0}.

\begin{HLSL}
  Buffer<float2> buf1 : register(t0, space0);
  Buffer<float2> buf2 : register(t1); // defaults to space0
\end{HLSL}

When the register is not specified, resources will recieve implementation-dependent register assignments.
