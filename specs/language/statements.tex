\Ch{Statements}{Stmt}
\Sec{Label Statements}{Stmt.Label}
\Sec{Attributes}{Stmt.Attr}
\Sub{Unroll Attribute}{Stmt.Attr.Unroll}
The \texttt{[unroll]} and \texttt{[unroll(n)]} attribute qualifiers can be used
 to specify that a loop (\texttt{for}, \texttt{while} and \texttt{do while} 
loops) can be unrolled. This attribute qualifier can be used to specify full 
unrolling or partial unrolling by a specified amount. This is a compiler hint 
and the compiler may ignore this directive.

\texttt{n} is the loop unrolling factor and must be a positive integral compile time 
constant expression. If n is not specified, the compiler determines the 
unrolling factor for the loop. This attribute is not compatible with the \texttt{[loop]}
 attribute.

Note:  The \texttt{[unroll]} attribute  must appear immediately before the loop
 to be affected.

\subsubsection{Unroll Examples}{Stmt.Attr.Unroll.Examples}
\begin{itemize}
\item  Requests the compiler to unroll the above \texttt{for} loop by a factor 
of 2.
\begin{HLSL}
[unroll(2)]
int sum = 0;
for (int i=0; i<32; i++)
    sum += i;
\end{HLSL}
\item In this example, the compiler will determine how much to unroll the loop.
\begin{HLSL}
int i = 0;
[unroll]
do {
    ...
    i++;
} while(i < 32);
\end{HLSL}
\item Below is an  examples of \textbf{\emph{invalid}} usage of 
\texttt{[unroll(n)]} because n is not positive
\begin{HLSL}
[unroll(-1)]
while (...)
{
    ...
}
\end{HLSL}
\item This example is \textbf{\emph{invalid}} because the \texttt{[unroll]} 
attribute is used on a non-loop construct
\begin{HLSL}
[unroll]
if (...)
{
    ...
}
\end{HLSL}
\item The below example is \textbf{\emph{invalid}} because the loop unroll 
factor is not a compile-time constant expression.
\begin{HLSL}
    int x;
    [unroll(x)]
    for (int i=0; i<x; i++)
    {
        ...
    }
\end{HLSL}
\end{itemize}
\Sub{Loop Attribute}{Stmt.Attr.Loop}
The Attribute \texttt{[loop]} tells the compiler to execute each iteration of 
the loop. In other words, its a hint to indicate a loop should not be 
unrolled. Therefore it is not compatible with the \texttt{[unroll]} attribute.

Note:  The \texttt{[loop]} attribute  must appear immediately before the loop
 to be affected.
 
\subsubsection{Loop Examples}{Stmt.Attr.Loop.Examples}
\begin{itemize}
\item  Tell the compiler to not to unroll the loop.
\begin{HLSL}
[loop]
for (int i=0; i<32; i++) {
    ...
}
\end{HLSL}
\item This example is \textbf{\emph{invalid}} because the \texttt{[loop]} 
attribute is used on a non-loop construct
\begin{HLSL}
[loop]
if (...)
{
    ...
}
\end{HLSL}
\end{itemize}