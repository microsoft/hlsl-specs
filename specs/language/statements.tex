\Ch{Statements}{Stmt}
\Sec{Statements Grammar}{Stmt.Grammar}
\begin{grammar}
    \define{statement}\br
    labeled-statement\br
    \opt{attribute-specifier-sequence} expression-statement\br
    \opt{attribute-specifier-sequence} compound-statement\br
    \opt{attribute-specifier-sequence} iteration-statement\br
    \opt{attribute-specifier-sequence} selection-statement\br
    declaration-statement
\end{grammar}
\Sec{Label Statements}{Stmt.Label}
\p The optional \textit{attribute-specifier-sequence} applies to the statement that immediately follows it.
\Sec{Attributes}{Stmt.Attr}
\Sub{Unroll Attribute}{Stmt.Attr.Unroll}
\p The \textit{unroll}  attribute is only valid when applied to 
\textit{iteration-statements}. It is used to indicate that 
\textit{iteration-statements} like \texttt{for}, \texttt{while} and 
\texttt{do while} can be unrolled. This attribute qualifier can be used to 
specify full unrolling or partial unrolling by a specified amount. This is a 
compiler hint  and the compiler may ignore this directive.

\p The unroll attribute may optionally have an unroll factor represented as a 
single argument \texttt{n}  that is an integer constant expression value 
greater than zero. If n is not  specified, the compiler determines the 
unrolling factor for the loop. The \textit{unroll} attribute can not be applied
 to the same \textit{iteration-statement} as the \textit{loop} attribute.

 \p Note: The \texttt{[unroll]} attribute  must appear immediately before the \textit{iteration-statements}
 to be affected.

\Sub{Loop Attribute}{Stmt.Attr.Loop}
\p The Attribute \texttt{[loop]} tells the compiler to execute each iteration of 
the loop. In other words, its a hint to indicate a loop should not be 
unrolled. Therefore it is not compatible with the \texttt{[unroll]} attribute.

\p Note: The \texttt{[loop]} attribute  must appear immediately before the loop
 to be affected.
 