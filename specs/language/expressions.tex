\Ch{Expressions}{Expr}

\p This chapter defines the formulations of expressions and the behavior of
operators when they are not overloaded. Only member operators may be
overloaded\footnote{This will change in the future, but this document assumes
current behavior.}. Operator overloading does not alter the rules for operators
defined by this standard.

\p An expression may also be an \textit{unevaluated operand} when it appears in
some contexts. An \textit{unevaluated operand} is a expression which is not
evaluated in the program\footnote{The operand to \texttt{sizeof(...)} is a good
example of an \textit{unevaluated operand}. In the code \texttt{sizeof(Foo())},
the call to \texttt{Foo()} is never evaluated in the program.}.

\p Whenever a \textit{glvalue} appears in an expression that expects a
\textit{prvalue}, a standard conversion sequence is applied based on the rules
in \ref{Conv}.

\Sec{Usual Arithmetic Conversions}{Expr.conv}
\p Binary operators for arithmetic and enumeration type require that both
operands are of a common type. When the types do not match the \textit{usual
arithmetic conversions} are applied to yield a common type. When \textit{usual
arithmetic conversions} are applied to vector operands they behave as
component-wise conversions (\ref{Conv.cwise}). The \textit{usual arithmetic
conversions} are:

\begin{itemize}
  \item If either operand is of scoped enumeration type no conversion is
  performed, and the expression is ill-formed if the types do not match.
  \item If either operand is a \texttt{vector<T,X>}, vector truncation or scalar
  extension is performed with the following rules:
  \begin{itemize}
    \item If both vectors are of the same length, no dimension conversion is
    required.
    \item If one operand is a vector and the other operand is a scalar, the
    scalar is extended to a vector via a Splat conversion (\ref{Conv.vsplat}) to
    match the length of the vector.
    \item Otherwise, if both operands are vectors of different lengths, the
    vector of longer length is truncated to match the length of the shorter
    vector (\ref{Conv.vtrunc}).
  \end{itemize}
  \item If either operand is of type \texttt{double} or \texttt{vector<double,
  X>}, the other operator shall be converted to match.
  \item Otherwise, if either operand is of type \texttt{float} or \texttt{vector<float,
  X>}, the other operand shall be converted to match.
  \item Otherwise, if either operand is of type \texttt{half} or \texttt{vector<half, X>},
  the other operand shall be converted to match.
  \item Otherwise, integer promotions are performed on each scalar or vector
  operand following the appropriate scalar or component-wise conversion
  (\ref{Conv}).
  \begin{itemize}
    \item If both operands are scalar or vector elements of signed or unsigned
    types, the operand of lesser integer conversion rank shall be converted to
    the type of the operand with greater rank.
    \item Otherwise, if both the operand of unsigned scalar or vector element
    type is of greater rank than the operand of signed scalar or vector element
    type, the signed operand is converted to the type of the unsigned operand.
    \item Otherwise, if the operand of signed scalar or vector element type is
    able to represent all values of the operand of unsigned scalar or vector
    element type, the unsigned operand is converted to the type of the signed
    operand.
    \item Otherwise, both operands are converted to a scalar or vector type of
    the unsigned integer type corresponding to the type of the operand with
    signed integer scalar or vector element type.
  \end{itemize}
\end{itemize}

\Sec{Primary Expressions}{Expr.Primary}

\begin{grammar}
  \define{primary-expression}\br
  literal\br
  \keyword{this}\br
  \terminal{(} expression \terminal{)}\br
  id-expression\br
\end{grammar}

\Sub{Literals}{Expr.Primary.Literal}

\p The type of a \textit{literal} is determined based on the grammar forms
specified in \ref{Lex.Literal.Kinds}.

\Sub{This}{Expr.Primary.This}

\p The keyword \keyword{this} names a reference to the implicit object of
non-static member functions. The \keyword{this} parameter is always a
\textit{prvalue} of non-\textit{cv-qualified}type.
\footnote{
  \href{https://github.com/microsoft/hlsl-specs/blob/main/proposals/0007-const-instance-methods.md}
  {HLSL Specs Proposal 0007} proposes adopting C++-like syntax and semantics for
  \textit{cv-qualified} \keyword{this} references.}

\p A \keyword{this} expression shall not appear outside the declaration of a
non-static member function.

\Sub{Parenthesis}{Expr.Primary.Paren}

\p An expression (\textit{E}) enclosed in parenthesis has the same type, result
and value category as \textit{E} without the enclosing parenthesis. A
parenthesized expression may be used in the same contexts with the same meaning
as the same non-parenthesized expression.

\Sub{Names}{Expr.Primary.ID}

\begin{note}
  The grammar and behaviors of this section are almost identical to C/C++ with
  some subtractions (notably lambdas and destructors).
\end{note}

\begin{grammar}
  \define{id-expression}\br
  unqualified-id\br
  qualified-id
\end{grammar}

\SubSub{Unqualified Identifiers}{Expr.Primary.ID.Unqual}

\begin{grammar}
  \define{unqualified-id}\br
  identifier\br
  operator-function-id\br
  conversion-function-id\br
  template-id\br
\end{grammar}

\SubSub{Qualified Identifiers}{Expr.Primary.ID.Qual}

\begin{grammar}
  \define{qualified-id}\br
  nested-name-specifier \opt{\keyword{template}} unqualified-id\br

  \define{nested-name-specifier}\br
  \terminal{::}\br
  type-name \terminal{::}\br
  namespace-name \terminal{::}\br
  nested-name-specifier identifier \terminal{::}\br
  nested-name-specifier \opt{\keyword{template}} simple-template-id \terminal{::}
\end{grammar}

\Sec{Postfix Expressions}{Expr.Post}

\begin{grammar}
  \define{postfix-expression}\br
  primary-expression\br
  % The [] characters on the two lines below should be inside \terminal, however
  % pandoc doesn't seem to like that.
  postfix-expression [ expression ]\br
  postfix-expression [ braced-init-list ]\br %
  postfix-expression \terminal{(} \opt{expression-list} \terminal{)}\br
  simple-type-specifier \terminal{(} \opt{expression-list} \terminal{)}\br
  typename-specifier \terminal{(} \opt{expression} \terminal{)}\br
  simple-type-specifier braced-init-list\br
  typename-specifier braced-init-list\br
  postfix-expression \terminal{.} \opt{\terminal{template}} id-expression\br
  postfix-expression \terminal{.} vector-swizzle-component-sequence\br
  postfix-expression \terminal{.} matrix-swizzle-component-sequence\br
  postfix-expression \terminal{++}\br
  postfix-expression \terminal{--}
\end{grammar}

\Sec{Subscript}{Expr.Post.Subscript}

\p A \textit{postfix-expression} followed by an expression in square brackets
(\texttt{[ ]}) is a subscript expression. In an array subscript expression of
the form \texttt{E1[E2]}, \texttt{E1} must either be a variable of array,
vector, or matrix of \texttt{T[]}, or an object of type \texttt{T} where
\texttt{T} provides an overloaded implementation of \texttt{operator[]}
(\ref{Overload}).\footnote{HLSL does not support the base address of a subscript
operator being the expression inside the braces, which is valid in C and C++.}

\p If the postfix expression \texttt{E1} is of vector type, the expression
\textit{E2} must be a value of integer type. If the value is known at compile
time to be outside the range \texttt{[0, N-1]} where \texttt{N} is the number of
elements in the vector, the program is ill-formed. If the value is outside the
range at runtime, the behavior is undefined.

\Sec{Swizzle Expressions}{Expr.Post.Swizzle}
\p A \textit{postfix-expression} followed by a dot (\texttt{.}) and a sequence
of one or more \textit{swizzle components} is a postfix expression. The
postfix expression before the dot is evaluated and must be of vector or matrix type.
If the postfix expression before the dot is an lvalue, the swizzle expression may
produce either an lvalue or a prvalue; otherwise it produces a prvalue.


\p If the postfix expression before the dot is
an lvalue and the \textit{swizzle-component-sequence} contains no repeated
components, the swizzle expression is an lvalue; otherwise it is a prvalue.

\Sub{Vector Swizzle}{Expr.Post.Swizzle.Vector}

\begin{grammar}
  \define{vector-swizzle-component-sequence}\br
  swizzle-component-sequence-rgba\br
  swizzle-component-sequence-xyzw\br

  \define{swizzle-component-sequence-rgba}\br
  swizzle-component-rgba\br
  swizzle-component-sequence-rgba swizzle-component-rgba\br

  \define{swizzle-component-sequence-xyzw}\br
  swizzle-component-xyzw\br
  swizzle-component-sequence-xyzw swizzle-component-xyzw\br

  \define{swizzle-component-rgba} \textnormal{one of}\br
  \terminal{r g b a}\br

  \define{swizzle-component-xyzw} \textnormal{one of}\br
  \terminal{x y z w}
\end{grammar}

\p A \textit{vector-swizzle-component-sequence} is a sequence of one or more swizzle
components of either the \textit{swizzle-component-rgba} or
\textit{swizzle-component-xyzw} forms; the two forms may not be mixed in the
same \textit{vector-swizzle-component-sequence}. The type of a swizzle expression is a
vector of the same element type as the postfix expression before the dot, with a
number of elements equal to the number of components in the \textit{vector-swizzle-component-sequence}. 

\p Vector swizzle components map to elements of the vector in the following way:

\begin{center}
  \begin{tabular}{|| c | c | c ||}
    \hline
    Element Index & RGBA component & XYZW component \\
    \hline
    0 & r & x \\
    1 & g & y \\
    2 & b & z \\
    3 & a & w \\
    \hline
  \end{tabular}
\end{center}

\p A program is ill-formed if any component in the \textit{swizzle-component-sequence}
refers to an element index that is out of range of the vector type of the
postfix expression before the dot.

\Sec{Matrix Swizzle}{Expr.Post.Swizzle.Matrix}
\begin{grammar}
  \define{matrix-swizzle-component-sequence}\br
  matrix-zero-indexed-swizzle-sequence\br
  matrix-one-indexed-swizzle-sequence\br

  \define{matrix-zero-indexed-swizzle-sequence}\br
  matrix-zero-indexed-swizzle\br
  matrix-zero-indexed-swizzle-sequence matrix-zero-indexed-swizzle\br

  \define{matrix-zero-indexed-swizzle}\br
  \terminal{\_m} zero-index-value\br

  \define{zero-index-value} \textnormal{one of}\br
  \terminal{0 1 2 3 }\br

  \define{matrix-one-indexed-swizzle-sequence}\br
  matrix-one-indexed-swizzle\br
  matrix-one-indexed-swizzle-sequence matrix-one-indexed-swizzle\br

  \define{matrix-one-indexed-swizzle}\br
  \terminal{\_m} one-index-value\br

  \define{one-index-value} \textnormal{one of}\br
  \terminal{1 2 3 4 }\br
\end{grammar}
\begin{itemize}
\item \textbf{.\_} (math-style): subsequent subscripts use \textbf{1-based}
 indexing for both row and column.
\item \textbf{.\_m} (memory-style): subsequent subscripts use \textbf{0-based}
 indexing for both row and column.
\end{itemize}

\p A \textit{matrix-swizzle-component-sequence} is a sequence of one but less
than or equal to four swizzle components of either the
\textit{matrix-zero-indexed-swizzle} or \textit{matrix-one-indexed-swizzle}
forms; the two forms may not be mixed in the same 
\textit{matrix-swizzle-component-sequence}. The type of a swizzle expression is
a vector of the same element type as the postfix expression before the dot,
with a number of elements equal to the number of components in the
\textit{matrix-swizzle-component-sequence}.

\p Matrix swizzle components map to elements of the matrix in the following way:

\begin{center}
  \begin{tabular}{|| c | c | c ||}
    \hline
    Element Index & 0 indexed component & 1 indexed component \\
    \hline
    0,0 & \_m00 & \_11 \\
    0,1 & \_m01 & \_12 \\
    0,2 & \_m02 & \_13 \\
    0,3 & \_m03 & \_14 \\
    1,0 & \_m10 & \_21 \\
    1,1 & \_m11 & \_22 \\
    1,2 & \_m12 & \_23 \\
    1,3 & \_m13 & \_24 \\
    2,0 & \_m20 & \_31 \\
    2,1 & \_m21 & \_32 \\
    2,2 & \_m22 & \_33 \\
    2,3 & \_m23 & \_34 \\
    3,0 & \_m30 & \_41 \\
    3,1 & \_m31 & \_42 \\
    3,2 & \_m32 & \_43 \\
    3,3 & \_m33 & \_44 \\
    \hline
  \end{tabular}
\end{center}

\Sec{Function Calls}{Expr.Post.Call}

\p A function call may be an \textit{ordinary function}, or a \textit{member
function}. In a function call to an \textit{ordinary function}, the
\textit{postfix-expression} must be an lvalue that refers to a function. In a
function call to a \textit{member function}, the \textit{postfix-expression}
will be an implicit or explicit class member access whose \textit{id-expression}
is a member function name.

\p When a function is called, each parameter shall be initialized with its
corresponding argument. The order in which parameters are initialized is
unspecified. \footnote{Today in DXC targeting DXIL matches the Microsoft C++ ABI
and evaluates argument expressions right-to-left, while SPIR-V generation
matches the Itanium ABI evaluating parameters left-to-right. There are good
arguments for unifying these behaviors, and arguments for keeping them
different.}

\p If the function is a non-static member function the \texttt{this} argument
shall be initialized to a reference to the object of the call as if casted by an
explicit cast expression to an lvalue reference of the type that the function is
declared as a member of.

\p Parameters are either \textit{input parameters}, \textit{output parameters},
or \textit{input/output parameters} as denoted in the called function's
declaration (\ref{Decl.Function}). For all types of parameters the argument
expressions are evaluated before the function call occurs.

\p \textit{Input parameters} are passed by-value into a function. If an argument
to an \textit{input parameter} is of constant-sized array type, the array is
copied to a temporary and the temporary value is converted to an address via
array-to-pointer decay. If an argument is an unsized array type, the array
lvalue directly decays via array-to-pointer decay. \footnote{This results in
\textit{input} parameters of unsized arrays being modifiable by a function.}

\p Arguments to \textit{output} and \textit{input/output parameters} must be
lvalues. \textit{Output parameters} are not initialized prior to the call; they
are passed as an uninitialized cxvalue (\ref{Basic.lval}). An \textit{output
parameter} is only initialized explicitly inside the called function. It is
undefined behavior to not explicitly initialize an \textit{output parameter}
before returning from the function in which it is defined. The cxvalue created
from an argument to an \textit{input/output parameter} is initialized through
copy-initialization from the lvalue argument expression. Overload resolution
shall occur on argument initialization as if the expression \texttt{T Param =
Arg} were evaluated. In both cases, the cxvalue shall have the type of the
parameter and the argument can be converted to that type through implicit or
explicit conversion.

\p If an argument to an \textit{output} or \textit{input/output parameter} is a
constant sized array, the array is copied to a temporary cxvalue following the
same rules for any other data type. If an argument to an \textit{output} or
\textit{input/output parameter} is an unsized array type, the array lvalue
directly decays via array-to-pointer decay. An argument of a constant sized
array of type \texttt{T[N]} can be converted to a cxvalue of an unsized array
of type \texttt{T[]} through array to pointer decay. An unsized array of type
\texttt{T[]}, cannot be implicitly converted to a a constant sized array of type
\texttt{T[N]}.

\p On expiration of the cxvalue, the value is assigned back to the argument
lvalue expression using a resolved assignment expression as if the expression
\texttt{Arg = Param} were written\footnote{The argument expression is not
re-evaluated after the call, so any side effects of the call occur only before
the call.}. The argument expression must be of a type or able to convert to a
type that has defined copy-initialization to and assignment from the parameter
type. The lifetime of the cxvalue begins at argument expression evaluation, and
ends after the function returns. A cxvalue argument is passed by-address to the
caller.

\p If the lvalue passed to an \textit{output} or \textit{input/output parameter}
does not alias any other parameter passed to that function, an implementation
may avoid the creation of excess temporaries by passing the address of the
lvalue instead of creating the cxvalue.

\p When a function is called, any parameter of object type must have completely
defined type, and any parameter of array of object type must have completely
defined element type.\footnote{HLSL \textit{output} and \textit{input/output
parameters} are passed by value, so they must also have complete type.} The
lifetime of a parameter ends on return of the function in which it is
defined.\footnote{As stated above cxvalue parameters are passed-by-address, so
the expiring parameter is the reference to the address, not the cxvalue. The
cxvalue expires in the caller.} Initialization and destruction of each
parameter occurs within the context of the calling function.

\p The value of a function call is the value returned by the called function.

\p A function call is an lvalue if the result type is an lvalue reference type;
otherwise it is a prvalue.

\p If a function call is a prvalue of object type, the type of the prvalue must
be complete.
