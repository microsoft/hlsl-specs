\Ch{Declarations}{Decl}
\Sec{Preamble}{Decl.Pre}
\p Declarations generally specify how names are to be interpreted. Declarations have the form
\begin{grammar}
  \define{declaration-seq}\br
  declaration\br
  declaration-seq declaration\br

  \define{declaration}\br
  name-declaration\br
  special-declaration\br

  \define{name-declaration}\br
  block-declaration\br
  function-definition\br
  template-declaration\br
  namespace-definition\br
  empty-declaration\br
  attribute-declaration\br
  cbuffer-declaration\br

  \define{special-declaration}\br
  export-declaration-group\br
  cbuffer-member-declaration\br

  \define{block-declaration}\br
  simple-declaration\br
  namespace-alias-definition\br
  using-declaration\br
  using-directive\br
  static\_assert-declaration\br
  alias-declaration\br
  opaque-enum-declaration\br

  \define{empty-declaration}\br
  \terminal{;}

\end{grammar}

\Sec{Specifiers}{Decl.Spec}
\Sub{General}{Decl.Spec.General}
\p The specifiers that can be used in a declaration are
\begin{grammar}
  \define{decl-specifier}\br
  \textit{function-specifier}\br
  ...
\end{grammar}

\Sub{Function specifiers}{Decl.Spec.Fct}

\p A \textit{function-specifier} can be used only in a function declaration.

\begin{grammar}
  \define{function-specifier}\br
  \texttt{export}\br
\end{grammar}

\p The \texttt{export} specifier denotes that the function has program linkage (\ref{Basic.Linkage.Program}).

\p The \texttt{export} specifier cannot be used on functions directly or indirectly within an unnamed namespace.

\p Functions with program linkage can also be specified in \textit{export-declaration-group} (\ref{Decl.Export}).

\p If a function is declared with an \texttt{export} specifier then all redeclarations of the same function must also use the \texttt{export} specifier or be part of \textit{export-declaration-group} (\ref{Decl.Export}).

\Sec{Declarators}{Decl.Decl}
\Sec{Initializers}{Decl.Init}

\p The process of initialization described in this section applies to all
initializers regardless of the context.

\begin{grammar}
  \define{initializer}\br
  brace-or-equal-initializer\br
  \terminal{(} expression-list \terminal{)}\br

  \define{brace-or-equal-initializer}\br
  \terminal{=} initializer-clause\br
  braced-init-list\br

  \define{initializer-clause}\br
  assignment-expression\br
  braced-init-list\br

  \define{braced-init-list}\br
  \terminal{\{} initializer-list \opt{\terminal{,}} \terminal{\}}\br
  \terminal{\{} \terminal{\}}\br

  \define{initializer-list}\br
  initializer-clause\br
  initializer-list \terminal{,} initializer-clause\br
\end{grammar}

\Sub{Aggregate Initialization}{Decl.Init.Agg}

\p An \textit{aggregate} is a vector, matrix, array, or class.

\p The subobjects of an aggregate have a defined order. For vectors and arrays
the order is increasing subscript order. For matrices it is increasing subscript
order with the subscript nesting such that in the notation
\texttt{Mat[M][N]}, the ordering is \(Mat[0][0]...Mat[0][N]...
Mat[M][0]...Mat[M][N]\). For classes the order is base class, followed by member
subobjects in declaration order.

\p A \textit{flattened ordering} of subobjects can be produced by performing a
depth-first traversal of the subobjects of an object following the defined
subobject ordering.

\p Each \textit{braced initializer list} is comprised of zero or more
\textit{initializer-clause} expressions, which is either another braced
initializer list or an expression which generates a value that either is or can
be implicitly converted to an rvalue. Each assignment-expression is an object,
which may be a scalar or aggregate type. A \textit{flattened initializer
sequence} is a sequence of expressions constructed by a depth-first traversal
over each assignment-expression in an initializer-list and performing a
depth-first traversal accessing each subobject of the assignment-expression.

\p If the target object is an array of unknown size, the object is assumed to
have \(m\) possible elements during parsing, where \(m>0\).

\p An initializer-list is a valid initializer if for each element
\(E_{n \bmod m}\) in the target object's flattened ordering there is a
corresponding expression \(E_n\) in the flattened initializer sequence, which
can be implicitly converted to the element's type. For arrays of unknown size,
the total number of expressions in the flattened initializer sequence must be a
multiple of the array's base element type.

\p An initializer-list is invalid if the flattened initializer sequence contains
more or fewer elements than the target object's flattened ordering, or if any
initializer \(I_n\) cannot be implicitly converted to the corresponding element
\(E_n\)'s type.

\Sec{Function Definitions}{Decl.Function}
\Sec{Attributes}{Decl.Attr}
\Sub{Semantic Annotations}{Decl.Attr.Semantic}
\Sub{Entry Attributes}{Decl.Attr.Entry}

\Sec{Export Declarations}{Decl.Export}

\p One or more functions with \textit{external linkage} can be also specified in the form of

\begin{grammar}
  \define{export-declaration-group}\br
  \texttt{export} \terminal{\{} \opt{function-declaration-seq} \terminal{\}}\br

  \define{function-declaration-seq}\br
  \textit{function-declaration} \opt{function-declaration-seq}
\end{grammar}

\p The \texttt{export} specifier denotes that every \textit{function-declaration} included in \textit{function-declaration-seq} has \textit{external linkage} (\ref{Basic.Linkage.External}).

\p The \textit{export-declaration-group} declaration cannot appear directly or indirectly within an unnamed namespace.

\p Functions with \textit{external linkage} can also be declared with an \texttt{export} specifier (\ref{Decl.Spec.Fct}).

\p If a function is part of an \textit{export-declaration-group} then all redeclarations of the same function must also be part on a \textit{export-declaration-group} or be declared with an \texttt{export} specifier (\ref{Decl.Spec.Fct}).

\Sec{Constant Buffer Declarations}{Decl.cbuffer}

\begin{grammar}
  \define{cbuffer-declaration}\br
  \terminal{cbuffer} name \opt{resource-binding} \terminal{\{}
    \opt{cbuffer-member-seq} \terminal {\}}\br

  \define{cbuffer-member-seq}\br
  cbuffer-member-declaration\br
  cbuffer-member-seq cbuffer-member-declaration\br

  \define{cbuffer-member-declaration}\br
  block-declaration\br
  function-definition\br
  template-declaration\br
  empty-declaration
\end{grammar}

\p A \textit{cbuffer declaration} is declared with the \texttt{cbuffer} keyword.
The name of the cbuffer declaration does not declare a name, and cannot be
referenced from within the translation unit, nor is it required to be unique.
Each cbuffer declaration refers to a unique constant buffer resource
(\ref{Resources.cnbuf}).

\p Declarations within a cbuffer declaration that declare names, declare their
names in the scope containing the cbuffer declaration. The cbuffer declaration
itself does not declare a declaration scope. A cbuffer declaration may not
contain a \textit{namespace-declaration} or \textit{cbuffer-declaration}.
\footnote{These declarations were previously allowed in HLSL reference compilers
but are not supported in this specification.}

\p Variable declarations with program storage duration
(\ref{Basic.Storage.Program}) in the cbuffer declaration are called
\textit{shader constants}. Shader constants are implicitly \texttt{const} and
cannot be modified in program code.\footnote{A future version of this
specification will likely disallow variable declarations with storage durations
other than program storage duration in cbuffer declarations.}
