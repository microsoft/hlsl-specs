\Ch{Declarations}{Decl}
\Sec{Preamble}{Decl.Pre}
\p Declarations generally specify how names are to be interpreted. Declarations have the form
\begin{grammar}
  \define{declaration-seq}\br
  \textit{declaration}\br
  \textit{declaration-seq declaration}

  \define{declaration}\br
  \textit{name-declaration}\br
  \textit{special-declaration}
  
  \define{name-declaration}\br
  ...

  \define{special-declaration}\br
  \textit{export-declaration-group}\br
  ...
\end{grammar}

\Sec{Specifiers}{Decl.Spec}
\Sub{General}{Decl.Spec.General}
\p The specifiers that can be used in a declaration are
\begin{grammar}
  \define{decl-specifier}\br
  \textit{function-specifier}\br
  ...
\end{grammar}

\Sub{Function specifiers}{Decl.Spec.Fct}

\p A \textit{function-specifier} can be used only in a function declaration.

\begin{grammar}
  \define{function-specifier}\br
  \texttt{export}\br
\end{grammar}

\p The \texttt{export} specifier denotes that the function has \textit{external linkage} (\ref{Decl.Linkage.External}).

\p The \texttt{export} specifier cannot be used on functions directly or indirectly within an unnamed namespace.

\p Functions with \textit{external linkage} can also be specified in \textit{export-declaration-group} (\ref{Decl.Export}).

\p If a function is declared with an \texttt{export} specifier then all redeclarations of the same function must also use the \texttt{export} specifier or be part of \textit{export-declaration-group} (\ref{Decl.Export}).

\Sec{Declarators}{Decl.Decl}
\Sec{Initializers}{Decl.Init}
\Sec{Function Definitions}{Decl.Function}
\Sec{Attributes}{Decl.Attr}
\Sub{Entry Attributes}{Decl.Attr.Entry}

\Sec{Export Declarations}{Decl.Export}

\p One or more functions with \textit{external linkage} can be also specified in the form of

\begin{grammar}
  \define{export-declaration-group}\br
  \texttt{export} \terminal{\{} \opt{function-declaration-seq} \terminal{\}}\br
  
  \define{function-declaration-seq}\br
  \textit{function-declaration} \opt{function-declaration-seq}
\end{grammar}

\p The \texttt{export} specifier denotes that every \textit{function-declaration} included in \textit{function-declaration-seq} has \textit{external linkage} (\ref{Decl.Linkage.External}).

\p The \textit{export-declaration-group} declaration cannot appear directly or indirectly within an unnamed namespace.

\p Functions with \textit{external linkage} can also be declared with an \texttt{export} specifier (\ref{Decl.Spec.Fct}).

\p If a function is part of an \textit{export-declaration-group} then all redeclarations of the same function must also be part on a \textit{export-declaration-group} or be declared with an \texttt{export} specifier (\ref{Decl.Spec.Fct}).

\Sec{Linkage}{Decl.Linkage}

\p An entity that denotes an object, reference, function, type, template, namespace, or value, may have a \textit{linkage}. If a name has \textit{linkage}, it refers to the same entity as the same name introduced by a declaration in another scope. If a variable, function, or another entity with the same name is declared in several scopes, but does not have sufficient \textit{linkage}, then several instances of the entity are generated.

\p There are three linkages recognized: \textit{external linkage}, \textit{internal linkage} and \textit{no linkage}.

\Sub{External Linkage}{Decl.Linkage.External}

\p Entities with \textit{external linkage} can be referred to from the scopes in the other translation units and enable linking between them.

\p The following entities in HLSL have \textit{external linkage}:
\begin{itemize}
  \item entry point functions
  \item functions marked with \texttt{export} keyword
  \item global variables that are not marked \texttt{static} or \texttt{groupshared} \footnote{These are not really linked with other translation units but rather their values are loaded indirectly based on cbuffer mapping.}
  \item static data members of classes or template classes
\end{itemize}

\p Linkage of functions (including template functions) that are not entry points or marked with \texttt{export} keyword is implementation dependent. \footnote{In DXC today functions that are not entry points or exported have \textit{internal linkage} by default. This can be overriden by \texttt{-default-linkage} compiler option.}

\Sub{Internal Linkage}{Decl.Linkage.Internal}

\p Entities with \textit{internal linkage} can be referred to from all scopes in the current translation unit.

\p The following entities in HLSL have \textit{internal linkage}:
\begin{itemize}
  \item global variables marked as \texttt{static} or \texttt{groupshared}
  \item all entities declared in an unnamed namespace or a namespace within an unnamed namespace
  \item enumerations
  \item classes or template classes, their member functions, and nested classes and enumerations
\end{itemize}

\Sub{No Linkage}{Decl.Linkage.NoLinkage}

\p An entity with \textit{no linkage} can be referred to only from the scope it is in.

\p Any of the following entites declared at function scope or block scopes derived from function scope have no linkage:
\begin{itemize}
  \item local variables
  \item local classes and their member functions
  \item other entities declared at function scope or block scopes derived from function scope that such as typedefs, enumerations, and enumerators
\end{itemize}
