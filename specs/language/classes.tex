\Ch{Classes}{Classes}
\Sec{Bit-field}{Classes.Bitfield}
\p A bit-field is a class member with an explicit size specified in bits. Bit-fields must be of an underflying integer type.
\begin{HLSL}
struct Foo {
  uint A : 10;
  bool B : 1;
};
\end{HLSL}
\p A bit-field can be unnamed.  Unnamed bit-fields cannot be accessed or initialized.
\begin{HLSL}
struct Foo {
  uint : 10;
};
\end{HLSL}
\p The explicit size of a named bit-field can be any number of bits greater than zero and less than or equal to the size of the specified type.
\p The explicit size of an unnamed bit-field can also be zero.
\p A bit-field cannot be a static member.
\Sec{Static Members}{Classes.Static}
\Sec{Conversions}{Classes.Conversions}
\p A glvalue of a named bit-field with underlying type \texttt{T} can be converted
to a cxvalue of type \texttt{T$^\prime$} if there exists an implicit conversion
from \texttt{T} to \texttt{T$^\prime$}.
\p A glvalue or a prvalue of a type \texttt{T} can be converted to a glvalue
of a named bit-field of type \texttt{T$^\prime$} if there exists an implicit
conversion from \texttt{T} to \texttt{T$^\prime$}.  After the conversion from
\texttt{T} to \texttt{T$^\prime$} the value is truncated to the size of the bit-field.
